% !TEX TS-program = pdflatex
% !TEX encoding = UTF-8 Unicode

% This is a simple template for a LaTeX document using the "article" class.
% See "book", "report", "letter" for other types of document.

\documentclass[11pt]{report} % use larger type; default would be 10pt

\usepackage[utf8]{inputenc} % set input encoding (not needed with XeLaTeX)

%%% Examples of Article customizations
% These packages are optional, depending whether you want the features they provide.
% See the LaTeX Companion or other references for full information.

%%% PAGE DIMENSIONS
\usepackage{geometry} % to change the page dimensions
\geometry{a4paper} % or letterpaper (US) or a5paper or....
% \geometry{margin=2in} % for example, change the margins to 2 inches all round
% \geometry{landscape} % set up the page for landscape
%   read geometry.pdf for detailed page layout information

\usepackage{graphicx} % support the \includegraphics command and options

% \usepackage[parfill]{parskip} % Activate to begin paragraphs with an empty line rather than an indent

%%% PACKAGES
\usepackage{booktabs} % for much better looking tables
\usepackage{array} % for better arrays (eg matrices) in maths
\usepackage{paralist} % very flexible & customisable lists (eg. enumerate/itemize, etc.)
\usepackage{verbatim} % adds environment for commenting out blocks of text & for better verbatim
\usepackage{subfig} % make it possible to include more than one captioned figure/table in a single float
% These packages are all incorporated in the memoir class to one degree or another...

%%% HEADERS & FOOTERS
\usepackage{fancyhdr} % This should be set AFTER setting up the page geometry
\pagestyle{fancy} % options: empty , plain , fancy
\renewcommand{\headrulewidth}{0pt} % customise the layout...
\lhead{}\chead{}\rhead{}
\lfoot{}\cfoot{\thepage}\rfoot{}

%%% SECTION TITLE APPEARANCE
\usepackage{sectsty}
\allsectionsfont{\sffamily\mdseries\upshape} % (See the fntguide.pdf for font help)
% (This matches ConTeXt defaults)

%%% ToC (table of contents) APPEARANCE
\usepackage[nottoc,notlof,notlot]{tocbibind} % Put the bibliography in the ToC
\usepackage[titles,subfigure]{tocloft} % Alter the style of the Table of Contents
\renewcommand{\cftsecfont}{\rmfamily\mdseries\upshape}
\renewcommand{\cftsecpagefont}{\rmfamily\mdseries\upshape} % No bold!

\usepackage[parfill]{parskip}

%%% END Article customizations



%%% The "real" document content comes below...

\title{Automatic Star-rating Generation from Text Reviews}
\author{
  Sang-Woo Jun\\
  \texttt{wjun@csail.mit.edu}
  \and
  Chong-U Lim\\
  \texttt{culim@csail.mit.edu}
  \and
  Pablo Ortiz\\
  \texttt{portiz@csail.mit.edu}
}
%\date{} % Activate to display a given date or no date (if empty),
         % otherwise the current date is printed 

\begin{document}
\maketitle

\tableofcontents

\newpage
\chapter{Introduction}
	\section{Motivation}

\chapter{Background}

\paragraph \indent 
Accurate customer feedback is much sought after by commercial enterprises. The information is quite valuable and affects a comapny's
bottom line by aiding in the design of new products/services or helping to fix those that have not been well-received. This being the 
case, many companies invite their customers to submit reviews for products they have used. The format of these reviews can vary widely
from the \'Like\' format used by Facebook to lengthy surveys. Some formats, of course, are more used more in practice than others.

\paragraph \indent
One well-established format for customer reviews is that of five-star ratings paired with short text reviews. Feedback in this form
allows a customer to both give an overall rating for a product/service and provide meaningful commentary about the product/service. A 
problem related to the accuracy of the reviews comes up when feedback is formatted this way. The same numbered star rating will hold
different meanings for different customers and vary wildly. Furthermore, it has been observed that some star ratings will not match
the corresponding sentiment seen in the accompanying text commentary. For a commercial enterprise looking to collect accurate feedback
from its customer reviews, this should be a matter of much concern.

\paragraph \indent
To acquire more accurate information from customer feedback, we propose the removal of five-star ratings from these review formats
entirely. Instead, we propose that the star ratings be generated in a uniform fashion from the sentiment of the text commentary.

\chapter{Design}
\section{User Reviews}
\subsection{Extraction from Google Play}
\subsection{Tagging}

\chapter{Results}

\chapter{Analysis}

\chapter{Conclusion}

\appendix
\chapter{Datasets}


\bibliographystyle{plain}
\bibliography{references}

\end{document}
