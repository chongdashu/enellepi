% !TEX TS-program = pdflatex
% !TEX encoding = UTF-8 Unicode

% This is a simple template for a LaTeX document using the "article" class.
% See "book", "report", "letter" for other types of document.

\documentclass[11pt]{report} % use larger type; default would be 10pt

\usepackage[utf8]{inputenc} % set input encoding (not needed with XeLaTeX)

%%% Examples of Article customizations
% These packages are optional, depending whether you want the features they provide.
% See the LaTeX Companion or other references for full information.

%%% PAGE DIMENSIONS
\usepackage{geometry} % to change the page dimensions
\geometry{a4paper} % or letterpaper (US) or a5paper or....
% \geometry{margin=2in} % for example, change the margins to 2 inches all round
% \geometry{landscape} % set up the page for landscape
%   read geometry.pdf for detailed page layout information

\usepackage{graphicx} % support the \includegraphics command and options

% \usepackage[parfill]{parskip} % Activate to begin paragraphs with an empty line rather than an indent

%%% PACKAGES
\usepackage{booktabs} % for much better looking tables
\usepackage{array} % for better arrays (eg matrices) in maths
\usepackage{paralist} % very flexible & customisable lists (eg. enumerate/itemize, etc.)
\usepackage{verbatim} % adds environment for commenting out blocks of text & for better verbatim
\usepackage{subfig} % make it possible to include more than one captioned figure/table in a single float
% These packages are all incorporated in the memoir class to one degree or another...

%%% HEADERS & FOOTERS
\usepackage{fancyhdr} % This should be set AFTER setting up the page geometry
\pagestyle{fancy} % options: empty , plain , fancy
\renewcommand{\headrulewidth}{0pt} % customise the layout...
\lhead{}\chead{}\rhead{}
\lfoot{}\cfoot{\thepage}\rfoot{}

%%% SECTION TITLE APPEARANCE
\usepackage{sectsty}
\allsectionsfont{\sffamily\mdseries\upshape} % (See the fntguide.pdf for font help)
% (This matches ConTeXt defaults)

%%% ToC (table of contents) APPEARANCE
\usepackage[nottoc,notlof,notlot]{tocbibind} % Put the bibliography in the ToC
\usepackage[titles,subfigure]{tocloft} % Alter the style of the Table of Contents
\renewcommand{\cftsecfont}{\rmfamily\mdseries\upshape}
\renewcommand{\cftsecpagefont}{\rmfamily\mdseries\upshape} % No bold!

\usepackage[parfill]{parskip} % Uncomment this to have no paragraph indentation

\usepackage{mathtools}

%%% END Article customizations



%%% The "real" document content comes below...

\title{Automatic Star-rating Generation from Text Reviews}
\author{
  Sang-Woo Jun\\
  \texttt{wjun@csail.mit.edu}
  \and
  Chong-U Lim\\
  \texttt{culim@csail.mit.edu}
  \and
  Pablo Ortiz\\
  \texttt{portiz@csail.mit.edu}
}
%\date{} % Activate to display a given date or no date (if empty),
         % otherwise the current date is printed 

\begin{document}
\maketitle

\tableofcontents

\newpage
\chapter{Introduction}
For our final project, we designed a system that generates five-star ratings for a product based on a review of that product. Ratings are generated by classifying the overall sentiment of a product review on the five-star scale using the PMI-IR algorithm. In the actual implementation of the system, we constrained the domain of product reviews to those of mobile applications on the Google Play App store. It was necessary to constrain the system to a particular problem domain as different problem domains will produce different sentiments for the same words. The system we created can serve as a proof of concept for generating five-star ratings based on product reviews for products and services of various kinds.

\section{Motivation}


Accurate customer feedback is much sought after by commercial enterprises. The information is quite valuable and affects a comapny's bottom line by aiding in the design of new products/services or helping to fix those that have not been well-received. This being the  case, many companies invite their customers to submit reviews for products they have used. The format of these reviews can vary widely from the ``Like" format used by Facebook to lengthy surveys. Some formats, of course, are used more in practice than others.


One well-established format for customer reviews is that of five-star ratings paired with short text reviews. Feedback in this form allows a customer to both give an overall rating for a product/service and provide meaningful commentary about the product/service. Unfortunately, a problem related to the accuracy of the reviews comes up when feedback is formatted this way. A particular star rating can mean completely different things to two different customers. For example, suppose there is one customer who sees the world in black and white and another customer who sees the world along the full five-star gradient. A one-star review from the first customer could mean anything between minor dislike and absolute hatred. The content of his/her text review would likely shed light on that. A one-star review from the second customer would definitely indicate absolute hatred. Though this example is quite extreme, it illustrates a flaw inherent to this system of obtaining feedback: some star ratings will not match the corresponding sentiment seen in the accompanying text commentary. For a commercial enterprise looking to collect accurate feedback from its customer reviews, this should be a matter of much concern.


To acquire more accurate information from customer feedback, we propose the removal of five-star ratings from these review formats entirely. Instead, we propose that the star ratings be generated in a uniform fashion from the sentiment of the text commentary. To  realize this proposal, we created a analyzes customer reviews of mobile applications from the Google Play store.

\chapter{Background}
	\section{Sentiment Analysis}

%%% BEGIN Chapter: Design
\chapter{Design}
\section{User Reviews}
\subsection{Extraction from Google Play}
\subsection{Tagging}
\subsection{Phrase Extraction}
\label{subsection:phrase_extraction}
Extending upon the work performed by Turney~\cite{Turney2001}, we used the following keywords to correspond to the each of the 5 stars 
of a customer review on the Google Play store. 
\subsection{Semantic Orientation}

In order to gain insight regarding the user's sentiment based on his or her review, we make use of the extracted phrases from 
Section~\ref{subsection:phrase_extraction} as a measure of a user's sentiment towards the application. The basis of extracting information from the phrases is derived from the mutual information of the words contained within the phrase, against known words that we identified as having either \textit{positive} or \textit{negative} sentiment polarity. The Pointwise Mutual Information (PMI) between two words\cite{church1990}, is defined with the following equation:

% \begin{math}...\end{math}
\begin{equation*} PMI(word_1, word_2) = \log_2 \left(\frac{Pr[word_1 \: and \: word_2]}{Pr[word_1] \times Pr[word_2]}\right) \end{equation*}

In the equation, the term $Pr[word]$ refers to the probability of the word appearing if we were to sample randomly from a text corpus, while  $Pr[word_1 \: and \: word_2]$ refers to the probability of both words appearing together. The ratio between p(word1 \& word2) and p(word1) p(word2) is thus a measure of the degree of statistical dependence between the words. The log of this ratio is the information gain from one of the words when the other is observed. The PMI score has an additional property in which it assigns a proportionally higher score to the words that appear infrequently in a given corpus, but with a higher likelihood of appearing together in the event they do\cite{Vargas2010}. The semantic orientation of a given phrase is thus given by the formula:

\begin{equation*} SO(phrase) = PMI(phrase, \verb|<positive>|) - PMI(phrase, \verb|<negative>|)\end{equation*}

In the formula, the terms \verb|positive| and \verb|negative| refer to words which are respectively associated with positive sentiments and negative sentiments. 

\subsubsection{PMI-IR}
In Turney's implementation, he made used of the the words \textit{"excellent"} and \textit{"negative"}. Also, he introduces a technique called PMI-IR (Information Retrieval) in order to estimate the PMI values by using hit counts returned from search engine queries. He estimates the semantic orientation using the formula:

\begin{equation*} PMI(word_1, word_2) = \log_2 \left(\frac{hits(phrase \: NEAR ``excellent") \times hits(``poor") ]}{hits(phrase \: NEAR ``poor") \times hits(``excellent") }\right) \end{equation*}

The term \textit{NEAR} in the equation above is a search query modifier which only returns results in which the items are located within a fixed number of words between each other. 

As a basis for our implementation, we used a similar PMI-IR estimation for semantic orientation of the extracted phrases, but using Google as the search engine of choice, and replacing the \textit{NEAR} operator with Google's corresponding \textit{AROUND(n)} operator (where \text{n} indicates the maximum range of words of which the two items in the search could differ by.)

However, based on our initial findings, we identified two problems with this approach. Firstly, automated queries to Google go against its Terms of Service Agreement. Despite attempts to decrease the likelihood of being detected such as by limiting our queries to a random interval between 10-15 seconds, we got blocked after about 80-100 calls. A survey of other alternative search engines either resulted in similar limits enforced (Bing), or deprecated APIs (Yahoo, Altavista). This made the process a slow and cumbersome endeavour.

The second problem was associated with the accuracy of the calculated results, which are outlined in greater detail in Chapter~\ref{chapter:results}. In short, the results failed to convincingly associate phrases with the correct polarity, partly due to the fact that the Google Search API does not actually return accurate values for the number of hits.

\subsubsection{Corpus-Derived PMI}
In order to find a method of calculating SO scores for phrases quickly, and without limit, we turned towards an offline SO estimator which uses the \textit{NLTK} \verb|movie_reviews| corpus. The corpus is a collection of user reviews from the \textit{IMDB Movie Database} , and is separated into sentences associated with \textit{positive} and \textit{negative} reviews. In our implementation, we first created a probability distribution of words which appear in the \text{positive} reviews and \text{negative} reviews separately. Next, we estimated the semantic orientation of a phrase using the formula:

\begin{equation*} SO(word) = \log_2 \left(\frac{Pr_{pos}[word]}{Pr_{neg}[word]}\right) \end{equation*}

Where the term $Pr_{pos}[word]$ is the probability of occurence of the word $word$ using the frequency distribution model acquired from the \textit{positive} corpus. The term $Pr_{pos}[word]$ naturally refers to the corresponding definition associated with the \textit{negative} corpus. We can easily extend this to a phrase with two words, such as ``very annoying" by calculating $SO(``very") + SO(``annoying")$. This allows us to attain values such as $SO(`excellent`")=1.2432$, $SO(``poor")=-0.9219$, $SO(``very \: annoying")=-0.3952$ and $SO(``very \: happy")=0.6716$. Thus, this gives us a nice form in which positive phrases are given positive values, and negative phrases are given negative values. Also, the magnitude of a value indicates a stroner association with the polarity.


%%% END Chapter: Design


\chapter{Results}
\label{chapter:results}

\chapter{Analysis}

\chapter{Conclusion}

\appendix
\chapter{Datasets}


\bibliographystyle{plain}
\bibliography{references}

\end{document}
